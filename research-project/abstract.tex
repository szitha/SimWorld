% Abstracts are usually written in English, with a version in your
% mother tongue underneath
\chapter*{Abstract} 
\addcontentsline{toc}{chapter}{Abstract}


% Don't change anything above this.
The applications of machine learning have created an opportunity to deal with complex problems currently encountered in radio astronomy data processing. Calibration is one of the most important data processing steps that are required to produce high dynamic range images. This process involves the determination of calibration parameters to correct the observed data. These parameters include instrumental as well as astronomical parameters. Many astronomers uses CASA to compute the gain calibration solutions. In this work we present applications of machine learning to first generation calibration (1GC). These applications uses data from the KAT-7 telescope sensors to perform calibration as opposed to calculating the solutions with CASA. These methods are computationally less expensive and accurate in predicting the 1GC solutions and antenna behaviour. This is multi-output regression model called $\textit{ZCal}$, based on Random forest, Decision trees, Extremely randomized trees and K-neares neighbor algorithms. After the testing and validation of our models, we observe that this model has learned to generalize and predict 1GC solutions with the rms error accuracy $\approx$ $0.01<Err_{amplitude}<0.09$ for gain amplitude, and $0.2<Err_{phase}<0.5$ for gain phase.  
% At a unviersity like Stellenbosch you *must* produce an abstract in Afrikaans for your masters.
% At AIMS you are encouraged to repeat the abstract in your mother tongue
% French, Igbo, Mlagasy, etc. just write it using LaTeX's special
% characters.
% Arabic students see the arabic.tex file for an example
% Amharic use openoffice and export from there and import a figure here.
% Where the words do not exist put the English work in italics, or use mathematical symbols.


% Do not change anything below this except for adding your
% signature (replace images/signature.png) and your name.
\vfill
\section*{Declaration}
I, the undersigned, hereby declare that the work contained in this research project is my original work, and that any work done by others or by myself previously has been acknowledged and referenced accordingly.

% Scan your signature into a small picture called 'signature.png' and insert it
% above your name and the date:
\includegraphics[height=2cm]{images/signature.png} \newline \hrule
% Your name must be in English Capitalisation with no comma, and the Family name comes last. 
% Do note the date below. It is called the "deadline".
Simphiwe Nhlanhla Zitha, 30 May 2018

