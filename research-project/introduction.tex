\chapter{Introduction}

The upcoming Square Kilometer Array (SKA) is expected to produce terabytes of data every hour. With this exponential growth of data, challenges for data calibration, reduction and analysis also increase, making it difficult for astronomers to manually process and analyse the data. Therefore, intelligent and automated systems are required  to overcome these challenges.

One of the main issues in radio astronomy is determining the quality of observational data. Astronomical signals are very weak  by the time they reaches the earth’s surface. Hence, they easily get corrupted by  atmospheric interferences, incorrect observational parameters(telescope locations, telescope pointing parameters), malfunctioning signal receivers, interference from terrestrial man-made radio sources and tracking inaccuracies \cite{taylor1999synthesis}. Therefore it is required to do proper corrections to the observational data before doing science. Radio astronomers spend considerable amount of time performing different preprocessing steps called calibration which involves the determination of different parameters to correct the received data. These parameters generally include instrumental as well as astronomical parameters. The general strategy for doing these corrections is to make use of a calibrator source. Calibrator sources are used to determine astronomical parameters for data corrections because they have known characteristics such as the brightness, shape, and spectrum \cite{taylor1999synthesis}. This process of calibration is iterative and time-consuming.

During science observations different  external parameters like atmospheric pressure, temperature  wind conditions, and relative humidity  are also collected through thousands of sensors attached to the telescopes and its adjoining instrumentation. The data coming from different sensors may provide information about the external conditions that may have corrupted the observed data. This piece of information is not always included in the conventional calibration steps. We propose to use machine learning methods to predict the calibration solutions looking at pointing and environmental sensor data. This is mainly motivated by the fact that calibration steps do corrections to data that is corrupted by environmental parameters.

In this project, we make use of the data from the Karoo Array Telescope (KAT-7), an array consisting of $7$ telescopes which is a precursor to the MeerKAT radio telescope. We look at pointing azimuth, elevation, scan, offset, temperature, wind speed, air pressure, relative humidity sensor data recorded during  observations with a calibrator source PKS1613-586 to generate the training and testing dataset. The overall generated dataset contains sensor data per telescope and calibration solutions for correcting the signal received by each telescope in horizontal polarization(h-pol)  and vertical polarization(v-pol). These calibrator solutions are calculated using one of the traditional astronomy software called CASA which is used for data calibration and imaging in radio astronomy.

