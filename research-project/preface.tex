\chapter*{Preface}

\section*{Chapter 1}
In this chapter we present the basic introduction to radio astronomy including its history; evolving from single dish to interferometry; the required instrumentation to observe the sky; the steps undertaken during data processing to correct for the true sky and the importance of the daily "big data" generated by the instruments.    
\section*{Chapter 2}
In this chapter we present a brief introduction to machine learning and learn its effectiveness in situations where deep and predictive insights need to be uncovered from data sets that are complex and fast changing. How to fine-tune hyper parameters per algorithm used to obtain better results and minimum processing time.  
\section*{Chapter 3}
In this chapter we describe the machine learning methods and procedure used for to approach our problem; approach to radio astronomy data processing tools,;dataset used for training and testing; how different multi-output predictors respond to complex data set and results.
\section*{Chapter 4}
This brings us to the discussion of the obtained results our our machine learning algorithms from the testing and validation data sets. 
\section*{Chapter 5}
Finally this brings us to the conclusion and future work planned to improve the performance of our algorithm.