\chapter{Conclusion}

The use of machine learning and telescope sensor data have paved a new way for development of calibration methods. We have used the telescope pointing and environmental sensor data to learn on the variability of the complex gain calibration solutions G for the calibrator PKS1613-586 as a function of time. We took advantage of the large amount of external data generated by a radio telescope during science observations, and such information not being included in the currently used traditional calibration softwares. The implementation of  ZCal algorithm is based on regression machine learning algorithms with the aim of predicting the calibration solutions and study each antenna behaviour. Here we have considered multiple observations and collected the observable quantities such as air temperature, relative humidity, wind speed, wind direction, air pressure and telescope pointings for each track of the calibrator PKS1613-586 in time. With each 1GC calibration solutions per observation from CASA, we constructed a matrix of training sample $n_L$ and testing sample $n_T$ to train the machine learning algorithms DT, RF, EXT, and KNN to be able to discern the patterns that relate complex gain  calibration solutions of a specific calibrator to external parameters.       

Since gain solutions are complex, we have developed/implemented  ZCal to learn on phase and amplitude to allow simplicity in during the training and also incorporate the differences or variation of amplitude and phase per antenna due to various effects. Each learning algorithm ran on the learning sample $N$ times and its error was estimated on the test sample. We have presented statistical framework to measure the accuracy of each multi-output regression model and our results are encouraging with an rms error of $\approx < 0.5$. Comparing the performances of these algorithms, the RF, EXT and KNN shown to be the best for our purpose. Further, we then applied these methods to validation observation data sets where we obtained accurate predictions of amplitude and phase solutions on RF, EXT and KNN models. We were able predict solutions approximately close to CASA with strong predictions based mostly on amplitude. This shows that our method has strongly learned and generalized on amplitude than phase solutions. This can simply imply that the environment and and the pointing is more correlated to the amplitude, whereas the phase effects is mostly correlated to the sky effects. ZCal can also be used to predict unexplained behaviour patterns in antennas. This information is not included in prior to 1GC and can be useful in further calibration techniques and antenna investigation.

The analysis of the algorithms have shown how well tree-based ensemble models can predict and learn to generalize. The error
analysis have shown that EXT methods work by decreasing variance while at the same time increasing bias. Once the randomization level is properly adjusted, the variance almost vanishes while bias only slightly increases with respect to standard trees \citep{geurts2006extremely}. The resulting models RF and EXT and KNN are thus proven to be best in learning complex problems such as calibration. These method  has many advantages over
existing approaches, i.e, they can be trained
and used in only seconds and hence provide substantial
speed-ups over other methods, secondly they perform non-linear and non parametric regression, which means that the method can
use orders-of-magnitude fewer models for the same level
of precision, while additionally attaining a more rigorous
appraisal of uncertainties for the predicted quantities \citep{bellinger2016fundamental}. However it has some limitations which led to model in highly learning the behaviour of the amplitude than phase. This is due to less training data from KAT-7 and the training only being done looking at antenna based parameters. 

In this thesis, we have introduced a new approach of finding the correlation between sensor data and calibration solutions for radio observations. The results obtained proves that machine learning  methods can be used as a tool for calibration. This proves that we can generate CASA-like solutions using sensor data as a function of time and machine learning with high accuracy.

In this study, we trained the model looking only at time dependent corrections. However in calibration steps, there is also a process called bandpass calibration, which focuses on correcting for the frequency dependent effects. These variations in frequency arises as a result of non-uniform filter passbands or other frequency-dependent effects in signal transmission. It is usually the case that these frequency-dependent effects vary on timescales much longer than the time-dependent effects \citep{editioncasa}. The most complete approach to bandpass calibration using CASA is to observe a strong line-free continuum source to find the bandpass characteristics (amplitude and phase) of each antenna-receiver combination \citep{editioncasa}. In future, we intend to study the correlation between the sensor data and the response of the bandpass to further improve the performance of our models in predicting highly accurate gain amplitude and phase solutions for calibration. This will also improve ZCal in learning to predict the correct amplitude, as it currently learned strongly the sinusoidal variation. In addition, more sensor data will be included such digitiser input and output power, spectrometer, lower noise amplifier sensor data.  









