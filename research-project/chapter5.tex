\chapter{Conclusion}

We have shown that the application of machine learning techniques to telescope sensor data opens a new avenue in the development of calibration methods. We have used the telescope sensor data to learn the variability of the complex gain calibration solutions for the calibrator PKS1613-586 as a function of time. We took advantage of the large amount of ancillary data generated by the telescope sensors during scientific observations, which is almost completely ignored in conventional calibration approaches. The implementation of the ZCal algorithm is based on regression machine learning algorithms, with the aim of predicting the calibration solutions and studying each antenna’s behaviour. We have considered multiple observations and collected the observable quantities such as air temperature, relative humidity, wind speed, wind direction, air pressure and telescope pointings for each track on the calibrator PKS1613-586. Using the 1GC calibration solutions obtained with CASA, we constructed a matrix of training sample $n_L$ and testing sample $n_T$ to train the machine learning algorithms decision tree, random forest, extremely randomised trees, and K-nearest neighbor to be able to discern the patterns that relate complex gain solutions of  to external parameters.

Since gain solutions are complex, we have implemented ZCal to learn on phase and amplitude separately.  Each learning algorithm ran on the learning sample N times and its error was estimated on the test sample. We presented a statistical framework to measure the accuracy of each multi-output regression model and our results are encouraging with an rms error of $\approx < 0.5 \mathrm{rad}$ during testing of our models using the testing data for gain amplitude and phase. Comparing the performance of these algorithms, the random forest, extremely randomized tree and K-nearest neighbours were shown to be the best for our purpose.

Applying these methods to the validation observation datasets test-1 and test-2 yielded mixed results. We observed that the environmental and the pointing sensor readings were more strongly correlated to the amplitude than phase. Consequently, the ability to predict gain-phase was overall poor; gain-amplitude prediction was accurate in some cases (capturing non-trivial behaviour such as occilations), and completely failed in others. With the benefit of physical intuition, the poor performance on phase solutions is hardly surprising -- we know that water vapour in the troposphere is a significant contributor to gain-phase, and we know that our sensors are not particularly sensitive to it. However, our methods were quite deliberately designed with no inputs from any physical priors -- the purpose of the project was to show that ML techniques can make available connections "blindly", without access to physical intuition. The accurate prediction of gain-amplitudes, in some cases, suggests that this is indeed feasible. It is not clear what caused the failed predictions, although we can always speculate on physical differences between observations that our sensors were not sensitive to. We can therefore expect that with access to a larger array of sensors, the ZCal approach will be able to make better gain predictions. In particular, telescopes working in the millimetre regime such as ALMA (where atmospheric effects are critical) are often equipped with water-vapour radiometers, which provide an independent measure of the atmospheric conditions. It would be very interesting to apply ZCal to such observations. The advantage of the ZCal approach is precisely in its lack of physics -- as long as there is *some* correlation between sensors data and gains, we can expect that correlation to be learned without resorting to physical models.









